\chapter{Thesis Preparation Rules}

The following notes have been produced for the guidance of research 
degree candidates in the presentation of their theses. This information is 
taken from the Postgraduate Handbook and Graduate School web pages. 
To download a PDF version of the Handbook, follow the link on the 
quick links panel of the Graduate School web page at:
\\
\\
http://www.liv.ac.uk/gradschool/thesisprep.htm
\\
\\
or open 
\\
\\
http://www.liv.ac.uk/sas/administration/pgrhandbook.pdf
\\
\\
in Internet 
Explorer.  All students, however, should ensure that they also consult 
their supervisor(s) about the presentation of their theses.

\section{Sources}
Candidates must state generally in the preface and specifically in the body 
of the thesis the sources from which their information is derived and the 
extent to which they have availed themselves of the work of others.

\section{Length}
The various Regulations require that a thesis should be as concise as 
possible. In no circumstances should a thesis of more than 100,000 words 
be submitted for PhD or MD (60,000 for MPhil), including the footnotes 
and appendices, unless written permission has been obtained from the 
candidate's supervisor, the Head of the Department and the Faculty Board 
concerned. It is recommended that the Head of Department seek the view 
of potential examiners before granting his/her approval.

\section{References}
References to published work should be given consistently in a format that 
is currently accepted in the field of work covered by the thesis. If in doubt, 
candidates should consult their supervisors about the best method.

\section{Number of copies required}
Three copies of the thesis and any supporting papers are normally required. 
Two copies should be deposited in the office of the Dean of the candidate's 
Faculty, who will make arrangements for the examination of the thesis. 
The candidate should retain one copy. However, if three Examiners are 
appointed, three copies must be deposited in the office of the Dean of the 
candidate's Faculty.
After the examination, the University will retain two copies in permanent 
bindings. One of these copies will be retained in the University Library. 
The candidate's supervisor will retain the other copy.

\section{Restrictions on access to theses}
An author may impose restrictions on access to theses and copying 
annually for up to five years, if the Head of Department endorses the 
author's statement that such restriction is necessary for good reasons, e.g. 
preparation for publication or a patent application. This will not prevent the 
publication of the Abstract. Permanent restriction is not permitted, nor does 
the University accept theses written under contracts of secrecy (see section 
18 above and the Note on Theses in the current edition of the University 
Calendar). 

\section{Presentation and layout}
In the following specification some of the requirements of BS 4821:1990 
have been adopted to ensure that doctoral theses conform to the standards 
expected by the British Library. Copies of the British Standard (now 
withdrawn from publication) may be consulted in the Sydney Jones and 
Harold Cohen Libraries. Authors' rights are protected under the 
University's agreement with the British Library.
(vii) Typing, printing and copying
Type must be uniform and clear in all copies, for both text and illustrations. 
The minimum height for capital letters is 2 mm and the minimum x-height 
(height of lower-case 'x') 1.5 mm. The main body of the text must be in 
black ink on white paper.
A personal computer with a printer of good quality (e.g. laser or inkjet) 
must be used to produce the first copy. Good, permanent photocopies on 
plain paper are acceptable for the second and third copies. Copies made by 
chemical means, which may fade, are not. The copier must be checked 
before use to ensure that it does not produce extraneous marks on the 
copies.

\section{Binding and lettering}
Theses may be presented for examination in either permanent or temporary 
bindings. 
\begin{description}
\item[Permanent binding]
The thesis to be bound in book form in a strong cloth of any suitable 
colour. Maximum thickness 65 mm (2.5"): if of greater thickness, two or 
more volumes per copy will be required. The binding of all volumes must 
be identical. The thesis should be bound in such a way that it can be 
opened fully for ease of microfilming. Final hardbinding is undertaken off 
university campus by SRJ Ltd in Liverpool (phone: 0151 709 1354). Visit 
their website: http://www.srjservices.co.uk/.
Lettering on permanent bindings to be in gold. Front cover: title of thesis. 
Spine: Top: degree. Middle: surname and initials. Bottom: year of 
submission.
\item[Temporary binding]
The thesis should be presented in such a way that the pages cannot be 
readily removed. The use of ring binders is therefore not permitted. The 
candidate's surname, initials, the date (month and year) and the degree to 
be shown on the outside front cover. Softbinding of initial submission can 
be undertaken by the university print unit.
After the thesis has been approved by the Examiners, two copies must be 
permanently bound as above and deposited with the Dean of the 
candidate's Faculty before arrangements for the conferment of the degree 
can be made.
\end{description}

\section{Title page}
(Centred) Title of thesis
Then `Thesis submitted in accordance with the requirements of the 
University of Liverpool for the degree of Doctor in Philosophy (or other 
degree as appropriate) by $<$full forenames and surname$>$.'
Then (centred) Date (month and year) with suitable line spacing.

\section{Table of contents}
The table of contents must show chapter headings and page numbers. All 
separate sections of the thesis, such as bibliography, lists of abbreviations, 
supporting papers, etc., must also be identified on the contents page.

\section{Abstract}
Each copy of the thesis must be accompanied by a separate copy of the 
Abstract indicating the aims of the investigation and the results achieved. 
For microfilming purposes it must:

\begin{itemize}
\item	Be typed or printed although good photocopies are acceptable;
\item	be not longer than can be accomplished by single-spaced type on 
one side of an A4 sheet (about 450 words);
\item	show the author and title of the thesis in the form of a heading.
\end{itemize}

\section{Paper}
A4 white bond paper of 70 to 100 $g/m^2$ weight must be used for both 
originals and photocopies, except for any endpapers which carry no text. If 
both sides of the paper are used for text, then:

\begin{itemize}
\item	Both sides must be used in both copies which are to be 
permanently bound;
\item	there must be little or no `show-through' - paper lighter than 80 
$g/m^2$ should not be used;
\item	the full binding margin of 40 mm must be allowed on the left 
side of odd pages and the right side of even pages - other 
margins must be 25mm minimum.
\end{itemize}

 Margins and line spacing
1$1 \over 2$ spacing is advised, but at least double line spacing should be used for 
text that contains many subscripts and superscripts. Quotations may be 
indented. Authors should check the text carefully for `widows and orphans' 
and make full use of all error-checking facilities.
\section{Page numbers}
Pages should be numbered consecutively and the position of page numbers 
(candidate's choice or as advised by the supervisor) should be consistent 
throughout.
\section{Footnotes}
Footnotes should be inserted at the foot of the relevant page in single line 
spacing. Smaller type may be used, if available. A line should be ruled 
between footnotes and the text. Footnotes should be numbered 
consecutively throughout the thesis.
\section{Diagrams, maps, illustrations and supporting material}
Diagrams, maps and illustrations should be placed as near to the relevant 
text as possible. If it is necessary to place illustrations in a separate volume, 
the binding must match that of the text. Photographs must be prints of good 
quality and adequate size. Identical and permanent prints of any 
monochrome or colour photographs used must be securely mounted in 
each copy of the thesis.
Published papers submitted in support of the thesis should be sewn in by 
the bookbinder as an appendix.
Essential material that cannot be sewn in (large charts, tapes, floppy discs, 
CDs, microfiches, etc.) must be placed securely in a pocket attached to the 
inside back cover of each copy by the bookbinder. Before submitting 
material that cannot be read without special facilities, candidates must 
satisfy themselves and their supervisors (a) that it is essential to include 
such material and (b) that the Examiners have ready access to such 
facilities. 

\section{Further advice}
The following publications, which can be consulted in the University 
Libraries, give advice on the preparation of theses and methods of 
bibliographical reference. Students are advised to purchase their own 
copies of their chosen manual. 

\subsection{For Humanities and Social Sciences}
\emph{MHRA Style Book}, Modern Humanities Research Association, London. 
MLA Style Sheet, Modern Language Association of America, 
Baltimore. 
\\
\\
Watson, G: \emph{Writing a thesis: A guide to long essays and dissertations}. 
Longman, 1987. 
Turabian, K L: A manual for writers of term papers, theses and 
dissertations. University of Chicago Press, Chicago, 1987. 

\subsection{For Sciences, Engineering and Medicine}
Barrass, R: \emph{Scientists must write: A guide to better writing for 
scientists, engineers and students}. Science Paperbacks, Chapman \& 
Hall, 1978. 
\\
\\
Booth, V: Communicating in science: \emph{Writing and speaking}. 
Cambridge University Press. 
\\
\\
Lindsay, D: \emph{Guide to scientific writing --- a manual for students and 
research workers}. Longman, London, 1990. 
\\
\\
Lock, S: \emph{Thorne's better medical writing}. Pitman, London, 1977. 
O'Connor, M and Woodford, F P: \emph{Writing scientific papers in English: 
An Else-Ciba Foundation guide for authors}. Elsevier, Amsterdam, 
1976. 
\subsection{For all candidates}
Stanisstreet, M: \emph{Writing your thesis: Suggestions for planning and 
writing theses and dissertations in science-based disciplines}. 
University of Liverpool Research Sub-Committee, 1988. 
\\
\\
Stanisstreet, M: \emph{Preparing for your viva: Suggestions for preparing for 
postgraduate vivas in science-based disciplines}. University of 
Liverpool Research Sub-Committee, 1988. 
\\
\\
These booklets were written by a scientist with scientists principally in 
mind, but much of the advice therein will benefit those in other disciplines. 
Copies are normally issued automatically to research students in science-
based departments. They may also be obtained on request, free of charge, 
from Faculty Offices and the Student \& Examinations Division, Senate 
House. 
