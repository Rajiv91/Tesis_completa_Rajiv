\chapter{Introducción}
\section{Introducción}

  Conocer la posición, orientación y movimiento de la cabeza son cuestiones de gran importancia ya que pueden ayudar a las personas a interactuar con las computadoras de una forma más natural a como se realiza actualmente, dando como resultado en útiles  aplicaciones como: video conferencias, controles o interfaces hombre-máquina especiales, aplicaciones de realidad virtual. Además, si se toma en cuenta hacia dónde está mirando la persona y el tiempo que lleva en cierta posición puede proporcionar información adicional que ayudaría a inferir si está muy atenta a lo que está observando o si la persona está durmiendo.\\
  Por medio del análisis de imágenes capturadas a personas y en combinación con algoritmos de aprendizaje automático es posible conocer la posición y  orientación de sus cabezas con respecto a un marco de referencia, la información obtenida puede ayudar a conocer qué es lo que están observando las personas, lo cual es el objetivo principal del presente trabajo de tesis, estimar la mirada de las personas en un plano enfrente de ellas.
  
  \section{Objetivo general}
  Detectar el rostro de las personas y estimar su pose relativa a un plano virtual situado enfrente de ellas, asociando regiones en dicho plano con información de textura de los rostros detectados y su posición. 
  
  \section{Objetivos específicos}
  \begin{itemize}
  	\item Generar una base de datos de rostros humanos con la información de
  	pose asociada.
  	\item Desarrollar el sistema para que funcione en un rango de distancia am-
  	plio entre la persona y la cámara.
  	\item Implementación de un algoritmo clasificador para asociar la región que
  	se observa con poses determinadas.
  \end{itemize}
  
  \section{Estructura de la tesis}